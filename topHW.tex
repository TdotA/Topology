\documentclass[11pt,a4paper]{article}
\usepackage[utf8]{inputenc}
\usepackage[legalpaper, portrait, margin=1in]{geometry}
\usepackage{amsmath}
\usepackage{amsthm}
\usepackage{xcolor}
\usepackage{amsfonts}
\usepackage{amssymb}
\usepackage{makeidx}
\usepackage{graphicx}

%opening
\title{Topology - HW}
\author{Todor Antic (89191001)}
\newtheorem{Ex}{Exercise}
\newtheorem{Sol}{Solution}


\begin{document}


\maketitle

\begin{Ex}
	By definition a subset $U \subset \mathbb{R}$ is open if it is a union of open intervals. Now suppose $f: \mathbb{R} \rightarrow \mathbb{R}$ is any function. Show that $f^{-1}(a,b)$ is open whenever $(a,b)$ is an open interval if and only if $f^{-1}(U)$ is open whenever $U$ is open.
\end{Ex}

\begin{Sol}
\end{Sol}

\noindent$\implies$: Suppose $f$ is a real function, and $f^{-1}(a,b)$ is open for every open interval $(a,b)$. Now if $U$ is open in $\mathbb{R}$, then $U = \bigcup (a_i,b_i)$.
Now we WTS $f^{-1}(\bigcup (a_i,b_i)) = \bigcup f^{-1}(a_i,b_i)$:

\begin{gather}\nonumber
f^{-1}(\bigcup (a_i,b_i)) = \{x \in \mathbb{R}| f(x) \in \bigcup(a_i,b_i)\} =\\ \nonumber \{x \in \mathbb{R}| \exists i \text{s.t.} f(x) \in (a_i, b_i)\} = \bigcup\{x \in \mathbb{R}| f(x) \in (a_i,b_i) \} = \bigcup f^{-1}(a_i,b_i)
\end{gather}
So $f^{-1}(U)$ is a union of open intervals, thus open. \\

\noindent $\impliedby$: By an entirely similar argument we can show that the converse holds.

\begin{Ex} Let X be a space and each of $B_1, B_2 , \dots , B_n$ is closed. Show that $B_1 \cup B_2$ is closed, show that $\emptyset$ is closed and $X$ is closed. Show that $\bigcap_{n=1}^\infty B_n$ is closed.
\end{Ex}

\begin{Sol}\end{Sol}
\noindent(1): $B_i$ is closed, then $B_i^c$ is open, and as union of open sets needs to be open, $B_1^c \cup B_2^c$ is open, hence it's complement is closed, so $B_1 \cup B_2$ is closed. \\

\noindent(2): $X$ is closed because $X^c = \emptyset$ is open by definition. $\emptyset$ is closed because $\emptyset^c = X$ and $X$ is open in $X$ by definition.\\ 

\noindent(3): $\bigcap_{n=1}^\infty B_n$ is complement of $\bigcup_{n=1}^\infty B_n^c$ which is an open set, so our set is closed.

\begin{Ex}
	Suppose $X$ is a space and $B \subset X$. Show $l(B)$ is closed if $\{x\}$ is closed for every $x \in X$. Show $B \cup l(B)$ is closed. Show that $\bar{B} = B \cup l(B)$. Show that B is closed if and only if $l(B) \subset B$
\end{Ex}

\begin{Sol} \end{Sol}

\noindent First let's show that closed set contains all of it's limit points. Let $X$ be a space and $S \subset X$ be closed. Let $x \in l(S)$, therefore $\forall U \in T_x \text{ s.t. } x \in U$ has the following property: $U \cap (S \setminus \{x\}) \neq \emptyset$. Now suppose $x \not\in S$, then $U := X \setminus S$ is open and $x\in U$, but $U \cap (S \setminus \{x\}) = \emptyset $, giving us a contradiction. \\ \\
Now we can show that $l(B)$ is closed if $\{x\} is closed \forall x \in X$. First we take $S := l(l(B))$ and show $S \subset l(B)$. Take $x \in S$, by definition $\forall \text{open} U$ contains $y \in l(B) \text{ s.t. } y \neq x$. Now $U \setminus \{x\}$ is an open neighborhood of $y$, so by applying definition of $l(B)$, we get $z \in ((U \setminus \{x\}) \cap B) \text{ s.t. } z \neq y$. As this holds for all open neighborhoods of $x$, we get that $x \in l(B)$, hence $ S = l(l(B)) \subset l(B) \implies l(B)$ is closed.  \\ \\ 
Now let's show that (1): $B \cup l(B) \subset \bar{B}$ and (2): $\bar{B} \subset B \cup l(B)$: \\
(1): Take $x \in B \cup l(B)$, if $x \in B$ then, $x \in \bar{B}$ by definition so let $x \in l(B)$, and suppose $x \not\in \bar{B}$ for contradiction. Now $x in \bar{B}^c$ which is an open set, hence $\exists \text{ open }U$ s.t. $x \in U \and U \cap \bar{B} = \emptyset$, even further $U \cap B = \emptyset$, giving us a contradiction. \\

\noindent(2): Take $x \in \bar{B}$, even further assume $x \in (\bar{B} \setminus B)$. Now let $x \not\in l(B)$, then $\exists U \text{ s.t. } U\cap B =\{x\}$ so $x \in B$. \\

\begin{Ex}
	Suppose $(X, d)$ is a metric space and $\tau_d$ is a topology generated by the open metric balls. Show that $U \in \tau_d$ if and only if $U$ is a union of open metric balls.
\end{Ex}

\begin{Sol} \end{Sol}
\noindent$\implies$: Suppose $U \in \tau_d$ is an open set, and assume it's nonempty. Now if $U = X$, $U$ is a union of all open balls in $\tau_d$ as $X \in \tau_d$ because $X$ is an open ball in $d$ for any metric $d$. Now if $U \neq X$, but $U in \tau_d$ that means that $U$ is an open ball or a union of open balls, as union of sets in $\tau_d$ is still in $\tau_d$.

$\impliedby$: If $U$ is a union of open metric balls, then each component of $U$ is in $\tau_d$, hence $U \in \tau_d$.

\begin{Ex}
s	Let $X = \mathbb{R}$ and $d_1(x,y) = |x-y|$ and $d_2(x,y) = 2|x-y|$ be metrics. Show that $\tau_{d1} = \tau_{d2}$. 
\end{Ex}

\begin{Sol} \end{Sol}
\noindent Let's show that open sets in $d_1$ and $d_2$ are the same. Take $U$ open in $d_1$, now $\forall x \in U \exists \varepsilon_1 \text{ s.t. }$ \\$B_{\varepsilon1}(x) \subset U$. Now to prove that $U$ is open in $d_2$ just take $\varepsilon_2 = \varepsilon_1/2$ and your "epsilon ball" is in $U$. 

\begin{Ex}
	Suppose $f: X \rightarrow Y$ is a continuous and $\lim_{n \rightarrow \infty}x_n = x$. Show $\lim_{n \rightarrow \infty}f(x_n) = f(x)$.
\end{Ex}
\begin{Sol} \end{Sol} 
\noindent Suppose $U \in T_x$ is an open set containing $x$, by definition of convergence $\exists N$ s.t. $[n > N] \implies x_n \in U$. Now, as $f$ is continuous, we get $f(U) \in T_y$ and $f(x) \in U$, now we want to find $N_2$ s.t. $[n > N] \implies f(x_n) \in f(U)$. So take $N_2 = N$ and plug it in. Now $[n > N_2] \implies x_n \in U \implies f(x_n) \in f(U)$, as we wanted. \\

\begin{Ex}
	Consider $\mathbb{R}$ with following weird topology. Declare $U \subset \mathbb{R}$ open if and only if $\mathbb{R} \setminus U$ is empty, finite or countably infinite(or if $U = \emptyset$). Show that this is a topology. Now let $Y = \{0,1\}$ be a set with discrete topology. Define $f: \mathbb{R} \rightarrow Y$ so that $f(x) = 0$ iff $x \le 0$. Is $f$ continuous? Suppose $x_n \rightarrow x$. Must $\{x_n\}$ be eventually constant? Is it true that $x_n \rightarrow x \implies f(x_n) \rightarrow f(x)$? 
\end{Ex}

\begin{Sol} \end{Sol}
\noindent First let's show that $T_\mathbb{R}$ is a topology. $\emptyset \in T_\mathbb{R}$ by definition and $\mathbb{R} \in T_mathbb{R}$ as $\mathbb{R} \setminus \mathbb{R} = \emptyset$, so yes it is a topology.\\ 
Now to prove $f$ is continuous. Let $U = \{0\}$, $U$ is closed in $\mathbb{R}$, but $U = f^{-1}(\{0\})$ which is open in $Y$, so $f$ isn't continuous. \\

\begin{Ex}
	Suppose $X$ is metrizable, $Y$ is any space and $f: X \rightarrow Y$ is a function. Show $f$ is continuous if and only if $x_n \rightarrow x \implies f(x_n) \rightarrow f(x)$.
\end{Ex}

\begin{Sol} \end{Sol}
\noindent $\implies$: 

\begin{Ex}
	Show that convergent sequences have unique limits in $T_2$ spaces.
\end{Ex}
\begin{Sol} \end{Sol}
\noindent Let $x_n$ be a convergent sequence and suppose $x_n \rightarrow x$ and $x_n \rightarrow y$ s.t. $x \neq y$. Now take $U,V \in \tau $ such that $x \in U$ and $y \in V$ but $U \cap V = \emptyset$. Now $x_n \in U$ infinitely often and $x_n \in V$ infinitely often, thus $x_n \in (U \cap V)$ infinitely often, contradicting properties of $T_2$ space.

\begin{Ex}
	Show that space $X$ is $T_1$ iff $\{x\}$ is closed $\forall x \in X$.
\end{Ex}

\begin{Sol} \end{Sol}
\noindent $\implies$: Suppose $X$ is $T_1$. Then $\forall y \in \{x\}^c$  $\exists U_y \in \tau$ s.t. $y \in U_y, x\not\in U_y$. Now $\bigcup U_y$ is open and is exactly the complement of $\{x\}$, so $\{x\}$ is closed.  

\noindent$\impliedby$: Suppose $\{x\}$ is closed $\forall x \in X$, then $\{x\}^c \in \tau$ and $y \in \{x\}^c, \forall y\neq x$ and $x \not\in \{x\}^c$ so space is $T_1$.

\begin{Ex}
Let $X=\{a,b\}$ with the following open sets:$\emptyset, \{a\}, \{a,b\}$. Is $X$ $T_0$? Is $X$ $T_1$? 
\end{Ex}

\begin{Sol} \end{Sol}
\noindent $X$ is $T_0$ as we can find $U \in \tau$, such that $a \in U$, but $b \not\in U$, just take $U = \{a\}$. On the other hand $X$ isn't $T_1$ as we can't find a set in $\tau$ containing $b$ that doesn't contain $a$. 

\begin{Ex}
	Find a space that is not $T_0$.
\end{Ex}

\begin{Sol} \end{Sol}
\noindent Just take any two point space with indiscrete topology. It is not $T_0$ as we can't get $U \in \tau$ containing just one point but not the other.

\begin{Ex}(
	Check that $h: \mathbb{R} \rightarrow \mathbb{R}$ defined so that $h(x) = 3x$ is a homeomorphism.
\end{Ex} 
\begin{Sol} \end{Sol}
\noindent First let's check if $h$ is continuous. As we're working in $\mathbb{R}$, all open sets are unions of open intervals, now if we take $U \in \tau$ in the codomain, and assume $U$ is of the form $\bigcup_{i \in I}(a_i, b_i)$, then $f^{-1}(U) = \bigcup_{i \in I}(\frac{a_i}{3}, \frac{b_i}{3})$, which is obviously open. By similar argument the invers is continuous and hence $h$ is homeomorphism. 

\begin{Ex}
	Show that $(A, \tau_A)$ is a space, assuming $A \subset X$ and $\tau_A$ is a subspace topology.
\end{Ex}

\begin{Sol} \end{Sol}
\noindent To show that $(A. \tau_A)$ is a space we need to show $A \in \tau_A$ and $\emptyset \in \tau_A$. To show this we have to find $U \in \tau_X$ such that $U \cap A = A$, and $V \in \tau_X$ such that $V \cap A = \emptyset$. Let's take $U = X$, and $V = \emptyset$, by definition $U,V$ satisfy our needs, so $(A , \tau_A)$ is a space with a subspace topology.

\begin{Ex}
	Suppose $(X, \tau_X)$ and $(Y, \tau_Y)$ are spaces, $f: X \rightarrow Y$ is a continuous and $A \subset X$. Let $(A, \tau_A)$ be the space with subspace topology. Show $f|A$ is continuous. 
\end{Ex}

\begin{Sol} \end{Sol}
\noindent Let $U \in \tau_Y$ and let $V \in \tau_X$ be $f^{-1}(U)$. Now by definition $(f|A)^{-1} = A \cap V$ and hence open in the subspace topology, so $f|A$ is continuous.

\begin{Ex}
	Is discrete topology in fact a topology? Is course topology in fact a topology?
\end{Ex}

\begin{Sol} \end{Sol}
\noindent (1): Discrete topology is a topology because $X \in 2^X$ and $\emptyset \in 2^X$ and $\tau_X = 2^X$. \\
\noindent (2): Course topology is a topology because by definition it only contains emppty set and the whole set.

\begin{Ex}
	Show that the space $(X, \tau_X)$ has the discrete topology if and only if $\{x\}$ is open in $X \forall x \in X$ if and only if $\forall (Y, \tau_Y)$, and all functions $f: X \rightarrow Y$ are continuous. 
\end{Ex}

\begin{Sol} \end{Sol}
\noindent $\implies$: Let X be a space with discrete topology, thus every $U \subset X$ is open, and $\{x\} \subset X$, hence $\{x\}$ is open. Now take $v \in \tau_Y$ and any $f : X \rightarrow Y$, now $f^{-1}(V) \subset X \implies f^{-1}(V) \in \tau_X$, so $f$ is continuous. \\

\noindent $\impliedby$: Suppose any function $f: X \rightarrow Y$ is continuous. That means that if $\forall y \in Y$, $\{y\} \in \tau_Y$, $f^{-1}(y) = \{x\}$ is open, so $\{x\} \in \tau_X$. But now, any $\bigcup_{i \in I}\{x_i\} \in \tau_X$, so any subset of $X$ is open, so $X$ has discrete topology.

\begin{Ex}
	Show that the space $(X, \tau_X)$ has course topology if and only if for all spaces $(Y, \tau_Y)$ and all functions $f: Y \rightarrow X$, $f$ is continuous. 
\end{Ex}

\begin{Sol} \end{Sol}
\noindent $\implies$: Assume $(X, \tau_X)$ has course topology, and let $(Y, \tau_Y)$ be any space and $f Y \rightarrow X$ be any function. Now to show that $f$ is continuous we need to check that $X$ and $\emptyset$ have open preimages, however this is easy as $f^{-1}(X) = Y$ and $f^{-1}(\emptyset) = \emptyset$ which are open by definition of a topology. \\
\noindent $\impliedby$: Suppose $(Y, \tau_Y), (X, \tau_X)$ are spaces and any $f: Y \rightarrow X$ is continuous. Suppose $X$ has a not course topology, then we can always construct an $f_i$ that sends a closed set in $Y$ to an open set in $X$, except for $X$ and $\emptyset$ as $f^{-1}(X) = Y$ by definition and $f^{-1}(\emptyset) = \emptyset$ which are open in $Y$, so $X$ has course topology.
\begin{Ex}
	If $(X, \tau_a)$ is finer than $(X, \tau_b)$, which of the following functions is guaranteed to be continuous? $idx : (X , \tau_a) \rightarrow (X, \tau_b)$ or $jdx : (X , \tau_b) \rightarrow (X, \tau_a)$? Is either guaranteed to be homeomorphism?
\end{Ex}

\begin{Sol} \end{Sol}
Assume, $\tau_b \subset \tau_a$, now this guarantees that $idx$ is continuous, but $jdx$ isn't continuous, and thus neither is a homeomorphism.

\begin{Ex}
	If $(X, \tau_a)$ is finer than $(X, \tau_b)$, which space is likely to have more convergent sequences?
\end{Ex}

\begin{Sol} \end{Sol}
\noindent Courser topology is likely to have more convergent sequences as all convergent sequences in $\tau_a$ converge in $\tau_b$, but the converse doesn't need to hold. 

\begin{Ex}
	Suppose $(X, \tau_X)$ is a space, $A \subset X$, and $(A, \tau_A)$ has the subspace topology. Show that if $F \subset A$, then $F$ is closed in $(A, \tau_A)$ if and only if $F = A \cap C$, and $C$ is closed in $X$.  
\end{Ex}

\begin{Sol} \end{Sol}
\noindent $\implies$: Let $F$ be closed in $A$, then $F^c \in \tau_A$ and $\exists U \in \tau_X$ such that $F^c = A \cap U$, so now $F = A \cap U^c$ and $U^c$ is closed in X exactly like we wanted.

\noindent $\impliedby$: Take $F = U \cap A$ for some closed $U \in X$, then $F^c = U^c \cap A$, and since $U^c \in \tau_X$ we know $F^c \in \tau_A$, and hence $F$ is closed in A.

\begin{Ex}
  $\mathbb{R}$ with usual topology is connected.
\end{Ex}
 
\begin{Sol} \end{Sol}
Let $A \subset \mathbb{R}$ be open, nonempty and $A \neq \mathbb{R}$, now we need to show $A$ is not closed. Let $a \in A$ and take $c \not \in A$, $a < c$. Now define a set $Z = \{ x| x \in \mathbb{R}, [a,x] \subset A \}$, and let $b = sup(Z)$. By definition of $Z$, $b \not \in Z$, but $b \in \bar{Z}$, so $b \in l(Z) $. Now, $b \not\in Z \implies b \not\in A$ and $b \in \bar{Z} \implies b \in \bar{A}$, so $l(A) \not \subset A$, hence A is closed.  

\begin{Ex}
	Suppose $X$ is a space and $A \subset X$ is connected. Show that $\bar{A}$ is connected.
\end{Ex}

\begin{Sol} \end{Sol}
\noindent Take $Y \subset X = \bar{A}$, now $A$ is dense in $Y$. Take nonempty $U \subset Y$ and assume $U$ is clopen in $Y$, now we want to show $U = Y$. By definition $U$ is clopen in $A$, further, either $U \cap A = A$ or $U \cap A = \emptyset$. To show that intersection is nonempty take $x \in U$, now if $x \not\in A$, then $x \in \bar{A} \implies x \in l(A)$. Now by definition of a limit point $\forall S \in \tau$ s.t. $x \in S$, $S \cap A \neq \emptyset$, and as $U \in \tau$, we can't have $U \cap A = \emptyset$, so $U \cap A = A$. Now: $$A \subset U \implies \bar{A} \subset \bar{U} \implies Y \subset \bar{U} \implies Y = U$$ As we wanted to show.

\begin{Ex}
	Suppose $X$ is a space and let $a \in X$. Now assume that $A_i$ is connected and $a \in A_i$ $\forall i \in I$. Show that if $\bigcup_{i \in I}A_i = X$, X is connected. 
\end{Ex}

\begin{Sol} \end{Sol}
\noindent Suppose $U \subset X$ is clopen, and $U \neq \emptyset$, now by definition, $U^c$ is clopen. Assume $a \in U$, now we want to prove $X \subset U$. To do this, let's get $x \in A_i$, note that $A_i \cap U$ is clopen and $a \in A_i \cap U$, hence $A_i \cap U = A_i$, so $x \in U$. Hence, $X = U$, $\forall \text{ clopen and nonempty } U \in X$, so $X$ is connected.

\begin{Ex}
	If $X$ is a space and $a \in X$, then there is a unique component $A \subset X$, such that $a \in A$.
\end{Ex}

\begin{Sol} \end{Sol}
\noindent Take $a \in X$ and let $A_i$ be the union of all connected subsets $B$ such that $a \in B$. Now $A_i$ is connected. Furthermore $A_i$ is maximal by it's definition. Now if $A$ and $B$ are components and $A \cap B \neq \emptyset$, then $A \cup B$ is connected, hence $A = B$. Thus the component containing any $x \in X$ is unique.

\begin{Ex}
	Suppose $(X, \tau_X)$ is connected and $f: (X, \tau_X) \rightarrow (Y, \tau_Y)$ is a continuous surjection. Show that $(Y, \tau_Y)$ is connected.
\end{Ex}

\begin{Sol} \end{Sol}
\noindent Take a clopen, nonempty $U \subset Y$ and show that $U = Y$. Assume opposite for contradiction, now $f^{-1}(U)$ and $(f^{-1}(U))^c$ are both open, but the only nonempty set satisfying that is $X$, so now $f^{-1}(U) = X$, and as $f$ is surjective that means that $U = Y$. So $Y$ is connected.

\begin{Ex}
	Suppose $X$ is a space and $A \subset X$ is a component of $X$. Why is $A$ closed?
\end{Ex}

\begin{Sol} \end{Sol}
\noindent By definition $A$ is a maximal connected set and is therefore clopen, and any clopen set is closed.

\begin{Ex}
	Suppose $X$ is a space and $A \subset X$ is closed, $B \subset X$ is closed and $f: A \rightarrow Y$ and $g: B \rightarrow Y$ are continuous such that $f|A \cap B = g|A\cap B$. Then $h = f \cup g: A\cup B \rightarrow Y$ is continuous.
\end{Ex}

\begin{Sol} \end{Sol}
\noindent Firstly, let $K \subset Y$ be closed, then $h^{-1}(K) = (h^{-1}(K) \cap A) \cup (h^{-1}(K) \cap B) = h^{-1}(K) \cap (A \cup B)$. So now $h^{-1}(K)$ is the union of two closed subsets of $X$. Thus by subspace topology, $h^{-1}$ is closed in $A \cup B$.

\begin{Ex}
	Suppose $(Y, \tau_Y)$ is a space and $X \subset Y$. Show that the following are equivalent. The subspace $(X, \tau_X)$ is compact. If $\{V_i\}$ is a collection of open sets in $\tau_Y$ covering  $X$, there exists finitely many sets $\{V_1, V_2, \dots ,V_n\} \subset \{V_i\}$ such that $V \subset \bigcup_{j=1}^n V_j$.

\end{Ex}

\begin{Sol} \end{Sol}
\noindent $\implies$: If $X$ is compact then every open cover yields a finite subcover. Now take $\{V_i\}$ to be the open cover and by definition it will yield a finite subcover, i.e $\{V_1, V_2, \dots ,V_n \}$. Converse is entirely similar. 

\begin{Ex}
	$[0,1]$ is compact using the subspace topology of $\mathbb{R}$.
\end{Ex}

\begin{Sol}\end{Sol}

\noindent Since every open $U$ in $\mathbb{R}$ is a union of open intervals, it suffices to prove the special case when $[0,1]$ is covered by open intervals. Suppose $[0,1] \subset \bigcup_{i \in I}(a_i,b_i)$. Now let $K = \{x \in [0,1]| [0,x] \text{ can be covered by finitely many }(a_i, b_i) \}$. $0 \in K \implies K \neq \emptyset$. If $x \in K$ and $0 < y < x$, then $y \in K$, and since $K \subset [0,1]$, we can express $K$ as either $[0,b]$ or $[0,b)$. Now let $b$ be $sup(K)$, so $b \in K$, now if we take $b<1$, we can always find another $a \in K$ such that $a > k$, contradicting maximality of $b$, hence $b = 1$, $K = [0,1]$, $[0,1]$ is compact.

\begin{Ex}
	Suppose $(X, \tau_X)$ is a compact space, and $f: (X, \tau_X) \rightarrow (Y,\tau_Y)$ is a continuous surjection. Show that $(Y, \tau_Y)$ is a compact space.  
\end{Ex}

\begin{Sol} \end{Sol}
Take a family  of open covers $O \in \tau_Y$ covering $Y$ and prove that it yields a finite subcover. Firstly note that $f^{-1}(O) \in \tau_X$ and thus yields a finite subcover, thus the image of that finite subcover is open in $Y$ and covers $Y$. so $Y$ is compact.

\begin{Ex}
	Supoose $(X , \tau_X)$ is a compact space and $A \subset X$ is closed. Prove that $A, \tau_A$ is compact.
\end{Ex}

\begin{Sol} \end{Sol}
Cover A by open sets from $X$ and throw in the open set $V = X \setminus A$, that gives us an open cover of $X$, now take the finite subcover guaranteed by compactness of $X$, and throw out $V$, notice we have covered $A$.

\end{document}