\documentclass[11pt,a4paper]{article}
\usepackage[utf8]{inputenc}
\usepackage[legalpaper, portrait, margin=1in]{geometry}
\usepackage{amsmath}
\usepackage{amsthm}
\usepackage{xcolor}
\usepackage{amsfonts}
\usepackage{amssymb}
\usepackage{makeidx}
\usepackage{graphicx}

%opening
\title{Topology - HW}
\author{Todor Antic (89191001)}
\newtheorem{Ex}{Exercise}
\newtheorem{Sol}{Solution}


\begin{document}


\maketitle

\begin{Ex}
	By definition a subset $U \subset \mathbb{R}$ is open if it is a union of open intervals. Now suppose $f: \mathbb{R} \rightarrow \mathbb{R}$ is any function. Show that $f^{-1}(a,b)$ is open whenever $(a,b)$ is an open interval if and only if $f^{-1}(U)$ is open whenever $U$ is open.
\end{Ex}

\begin{Sol}
\end{Sol}

\noindent$\implies$: Suppose $f$ is a real function, and $f^{-1}(a,b)$ is open for every open interval $(a,b)$. Now if $U$ is open in $\mathbb{R}$, then $U = \bigcup (a_i,b_i)$.
Now we WTS $f^{-1}(\bigcup (a_i,b_i)) = \bigcup f^{-1}(a_i,b_i)$:

\begin{gather}\nonumber
f^{-1}(\bigcup (a_i,b_i)) = \{x \in \mathbb{R}| f(x) \in \bigcup(a_i,b_i)\} =\\ \nonumber \{x \in \mathbb{R}| \exists i \text{s.t.} f(x) \in (a_i, b_i)\} = \bigcup\{x \in \mathbb{R}| f(x) \in (a_i,b_i) \} = \bigcup f^{-1}(a_i,b_i)
\end{gather}
So $f^{-1}(U)$ is a union of open intervals, thus open. \\

\noindent $\impliedby$: By an entirely similar argument we can show that the converse holds.

\begin{Ex} Let X be a space and each of $B_1, B_2 , \dots , B_n$ is closed. Show that $B_1 \cup B_2$ is closed, show that $\emptyset$ is closed and $X$ is closed. Show that $\bigcap_{n=1}^\infty B_n$ is closed.
\end{Ex}

\begin{Sol}\end{Sol}
\noindent(1): $B_i$ is closed, then $B_i^c$ is open, and as union of open sets needs to be open, $B_1^c \cup B_2^c$ is open, hence it's complement is closed, so $B_1 \cup B_2$ is closed. \\

\noindent(2): $X$ is closed because $X^c = \emptyset$ is open by definition. $\emptyset$ is closed because $\emptyset^c = X$ and $X$ is open in $X$ by definition.\\ 

\noindent(3): $\bigcap_{n=1}^\infty B_n$ is complement of $\bigcup_{n=1}^\infty B_n^c$ which is an open set, so our set is closed.

\begin{Ex}
	Suppose $X$ is a space and $B \subset X$. Show $l(B)$ is closed if $\{x\}$ is closed for every $x \in X$. Show $B \cup l(B)$ is closed. Show that $\bar{B} = B \cup l(B)$. Show that B is closed if and only if $l(B) \subset B$
\end{Ex}

\begin{Sol} \end{Sol}

\noindent First let's show that closed set contains all of it's limit points. Let $X$ be a space and $S \subset X$ be closed. Let $x \in l(S)$, therefore $\forall U \in T_x \text{ s.t. } x \in U$ has the following property: $U \cap (S \setminus \{x\}) \neq \emptyset$. Now suppose $x \not\in S$, then $U := X \setminus S$ is open and $x\in U$, but $U \cap (S \setminus \{x\}) = \emptyset $, giving us a contradiction. \\ \\
Now we can show that $l(B)$ is closed if $\{x\} is closed \forall x \in X$. First we take $S := l(l(B))$ and show $S \subset l(B)$. Take $x \in S$, by definition $\forall \text{open} U$ contains $y \in l(B) \text{ s.t. } y \neq x$. Now $U \setminus \{x\}$ is an open neighborhood of $y$, so by applying definition of $l(B)$, we get $z \in ((U \setminus \{x\}) \cap B) \text{ s.t. } z \neq y$. As this holds for all open neighborhoods of $x$, we get that $x \in l(B)$, hence $ S = l(l(B)) \subset l(B) \implies l(B)$ is closed.  \\ \\ 
Now let's show that (1): $B \cup l(B) \subset \bar{B}$ and (2): $\bar{B} \subset B \cup l(B)$: \\
(1): Take $x \in B \cup l(B)$, if $x \in B$ then, $x \in \bar{B}$ by definition so let $x \in l(B)$, and suppose $x \not\in \bar{B}$ for contradiction. Now $x in \bar{B}^c$ which is an open set, hence $\exists \text{ open }U$ s.t. $x \in U \and U \cap \bar{B} = \emptyset$, even further $U \cap B = \emptyset$, giving us a contradiction. \\

\noindent(2): Take $x \in \bar{B}$, even further assume $x \in (\bar{B} \setminus B)$. Now let $x \not\in l(B)$, then $\exists U \text{ s.t. } U\cap B =\{x\}$ so $x \in B$. \\

\begin{Ex}
	Suppose $(X, d)$ is a metric space and $\tau_d$ is a topology generated by the open metric balls. Show that $U \in \tau_d$ if and only if $U$ is a union of open metric balls.
\end{Ex}

\begin{Sol} \end{Sol}
\noindent$\implies$: Suppose $U \in \tau_d$ is an open set, and assume it's nonempty. Now if $U = X$, $U$ is a union of all open balls in $\tau_d$ as $X \in \tau_d$ because $X$ is an open ball in $d$ for any metric $d$. Now if $U \neq X$, but $U in \tau_d$ that means that $U$ is an open ball or a union of open balls, as union of sets in $\tau_d$ is still in $\tau_d$.

$\impliedby$: If $U$ is a union of open metric balls, then each component of $U$ is in $\tau_d$, hence $U \in \tau_d$.

\begin{Ex}
s	Let $X = \mathbb{R}$ and $d_1(x,y) = |x-y|$ and $d_2(x,y) = 2|x-y|$ be metrics. Show that $\tau_{d1} = \tau_{d2}$. 
\end{Ex}

\begin{Sol} \end{Sol}
\noindent Let's show that open sets in $d_1$ and $d_2$ are the same. Take $U$ open in $d_1$, now $\forall x \in U \exists \varepsilon_1 \text{ s.t. }$ \\$B_{\varepsilon1}(x) \subset U$. Now to prove that $U$ is open in $d_2$ just take $\varepsilon_2 = \varepsilon_1/2$ and your "epsilon ball" is in $U$. 

\begin{Ex}
	Suppose $f: X \rightarrow Y$ is a continuous and $\lim_{n \rightarrow \infty}x_n = x$. Show $\lim_{n \rightarrow \infty}f(x_n) = f(x)$.
\end{Ex}
\begin{Sol} \end{Sol} 
\noindent Suppose $U \in T_x$ is an open set containing $x$, by definition of convergence $\exists N$ s.t. $[n > N] \implies x_n \in U$. Now, as $f$ is continuous, we get $f(U) \in T_y$ and $f(x) \in U$, now we want to find $N_2$ s.t. $[n > N] \implies f(x_n) \in f(U)$. So take $N_2 = N$ and plug it in. Now $[n > N_2] \implies x_n \in U \implies f(x_n) \in f(U)$, as we wanted. \\

\begin{Ex}
	Consider $\mathbb{R}$ with following weird topology. Declare $U \subset \mathbb{R}$ open if and only if $\mathbb{R} \setminus U$ is empty, finite or countably infinite(or if $U = \emptyset$). Show that this is a topology. Now let $Y = \{0,1\}$ be a set with discrete topology. Define $f: \mathbb{R} \rightarrow Y$ so that $f(x) = 0$ iff $x \le 0$. Is $f$ continuous? Suppose $x_n \rightarrow x$. Must $\{x_n\}$ be eventually constant? Is it true that $x_n \rightarrow x \implies f(x_n) \rightarrow f(x)$? 
\end{Ex}

\begin{Sol} \end{Sol}
\noindent First let's show that $T_\mathbb{R}$ is a topology. $\emptyset \in T_\mathbb{R}$ by definition and $\mathbb{R} \in T_mathbb{R}$ as $\mathbb{R} \setminus \mathbb{R} = \emptyset$, so yes it is a topology.\\ 
Now to prove $f$ is continuous. Let $U = \{0\}$, $U$ is closed in $\mathbb{R}$, but $U = f^{-1}(\{0\})$ which is open in $Y$, so $f$ isn't continuous. \\

\begin{Ex}
	Suppose $X$ is metrizable, $Y$ is any space and $f: X \rightarrow Y$ is a function. Show $f$ is continuous if and only if $x_n \rightarrow x \implies f(x_n) \rightarrow f(x)$.
\end{Ex}

\begin{Sol} \end{Sol}
\noindent $\implies$: 

\begin{Ex}
	Show that convergent sequences have unique limits in $T_2$ spaces.
\end{Ex}
\begin{Sol} \end{Sol}
\noindent Let $x_n$ be a convergent sequence and suppose $x_n \rightarrow x$ and $x_n \rightarrow y$ s.t. $x \neq y$. Now take $U,V \in \tau $ such that $x \in U$ and $y \in V$ but $U \cap V = \emptyset$. Now $x_n \in U$ infinitely often and $x_n \in V$ infinitely often, thus $x_n \in (U \cap V)$ infinitely often, contradicting properties of $T_2$ space.

\begin{Ex}
	Show that space $X$ is $T_1$ iff $\{x\}$ is closed $\forall x \in X$.
\end{Ex}

\begin{Sol} \end{Sol}
\noindent $\implies$: Suppose $X$ is $T_1$. Then $\forall y \in \{x\}^c$  $\exists U_y \in \tau$ s.t. $y \in U_y, x\not\in U_y$. Now $\bigcup U_y$ is open and is exactly the complement of $\{x\}$, so $\{x\}$ is closed.  

\noindent$\impliedby$: Suppose $\{x\}$ is closed $\forall x \in X$, then $\{x\}^c \in \tau$ and $y \in \{x\}^c, \forall y\neq x$ and $x \not\in \{x\}^c$ so space is $T_1$.

\begin{Ex}
Let $X=\{a,b\}$ with the following open sets:$\emptyset, \{a\}, \{a,b\}$. Is $X$ $T_0$? Is $X$ $T_1$? 
\end{Ex}

\begin{Sol} \end{Sol}
\noindent $X$ is $T_0$ as we can find $U \in \tau$, such that $a \in U$, but $b \not\in U$, just take $U = \{a\}$. On the other hand $X$ isn't $T_1$ as we can't find a set in $\tau$ containing $b$ that doesn't contain $a$. 

\begin{Ex}
	Find a space that is not $T_0$.
\end{Ex}

\begin{Sol} \end{Sol}
\noindent Just take any two point space with indiscrete topology. It is not $T_0$ as we can't get $U \in \tau$ containing just one point but not the other.

\begin{Ex}(
	Check that $h: \mathbb{R} \rightarrow \mathbb{R}$ defined so that $h(x) = 3x$ is a homeomorphism.
\end{Ex} 
\begin{Sol} \end{Sol}
First let's check if $h$ is continuous. 
\end{document}