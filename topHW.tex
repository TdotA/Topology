\documentclass[11pt,a4paper]{article}
\usepackage[utf8]{inputenc}
\usepackage[legalpaper, portrait, margin=1in]{geometry}
\usepackage{amsmath}
\usepackage{amsthm}
\usepackage{xcolor}
\usepackage{amsfonts}
\usepackage{amssymb}
\usepackage{makeidx}
\usepackage{graphicx}

%opening
\title{Topology - HW}
\author{Todor Antic (89191001)}
\newtheorem{Ex}{Exercise}
\newtheorem{Sol}{Solution}

\newcommand{\R}{\mathbb{R}}
\newcommand{\ra}{\rightarrow}
\begin{document}


\maketitle

\begin{Ex}
	By definition a subset $U \subset \mathbb{R}$ is open if it is a union of open intervals. Now suppose $f: \mathbb{R} \rightarrow \mathbb{R}$ is any function. Show that $f^{-1}(a,b)$ is open whenever $(a,b)$ is an open interval if and only if $f^{-1}(U)$ is open whenever $U$ is open.
\end{Ex}

\begin{Sol}
\end{Sol}

\noindent$\implies$: Suppose $f$ is a real function, and $f^{-1}(a,b)$ is open for every open interval $(a,b)$. Now if $U$ is open in $\mathbb{R}$, then $U = \bigcup (a_i,b_i)$.
Now we WTS $f^{-1}(\bigcup (a_i,b_i)) = \bigcup f^{-1}(a_i,b_i)$:

\begin{gather}\nonumber
f^{-1}(\bigcup (a_i,b_i)) = \{x \in \mathbb{R}| f(x) \in \bigcup(a_i,b_i)\} =\\ \nonumber \{x \in \mathbb{R}| \exists i \text{s.t.} f(x) \in (a_i, b_i)\} = \bigcup\{x \in \mathbb{R}| f(x) \in (a_i,b_i) \} = \bigcup f^{-1}(a_i,b_i)
\end{gather}
So $f^{-1}(U)$ is a union of open intervals, thus open. \\

\noindent $\impliedby$: By an entirely similar argument we can show that the converse holds.

\begin{Ex} Let X be a space and each of $B_1, B_2 , \dots , B_n$ is closed. Show that $B_1 \cup B_2$ is closed, show that $\emptyset$ is closed and $X$ is closed. Show that $\bigcap_{n=1}^\infty B_n$ is closed.
\end{Ex}

\begin{Sol}\end{Sol}
\noindent(1): $B_i$ is closed, then $B_i^c$ is open, and as union of open sets needs to be open, $B_1^c \cup B_2^c$ is open, hence it's complement is closed, so $B_1 \cup B_2$ is closed. \\

\noindent(2): $X$ is closed because $X^c = \emptyset$ is open by definition. $\emptyset$ is closed because $\emptyset^c = X$ and $X$ is open in $X$ by definition.\\ 

\noindent(3): $\bigcap_{n=1}^\infty B_n$ is complement of $\bigcup_{n=1}^\infty B_n^c$ which is an open set, so our set is closed.

\begin{Ex}
	Is $\bar{B}$ closed? Why or why not?
\end{Ex}
\begin{Sol}\end{Sol}
\noindent $\bar{B}$ is cearly closed as it is intersection of closed sets, and is by definition the smallest closed set containing $B$.
\begin{Ex}
	Suppose $X$ is a space and $B \subset X$. Show $l(B)$ is closed if $\{x\}$ is closed for every $x \in X$. Show $B \cup l(B)$ is closed. Show that $\bar{B} = B \cup l(B)$. Show that B is closed if and only if $l(B) \subset B$
\end{Ex}

\begin{Sol} \end{Sol}

\noindent First let's show that closed set contains all of it's limit points. Let $X$ be a space and $S \subset X$ be closed. Let $x \in l(S)$, therefore $\forall U \in T_x \text{ s.t. } x \in U$ has the following property: $U \cap (S \setminus \{x\}) \neq \emptyset$. Now suppose $x \not\in S$, then $U := X \setminus S$ is open and $x\in U$, but $U \cap (S \setminus \{x\}) = \emptyset $, giving us a contradiction. \\ \\
Now we can show that $l(B)$ is closed if $\{x\} is closed \forall x \in X$. First we take $S := l(l(B))$ and show $S \subset l(B)$. Take $x \in S$, by definition $\forall \text{open} U$ contains $y \in l(B) \text{ s.t. } y \neq x$. Now $U \setminus \{x\}$ is an open neighborhood of $y$, so by applying definition of $l(B)$, we get $z \in ((U \setminus \{x\}) \cap B) \text{ s.t. } z \neq y$. As this holds for all open neighborhoods of $x$, we get that $x \in l(B)$, hence $ S = l(l(B)) \subset l(B) \implies l(B)$ is closed.  \\ \\ 
Now let's show that (1): $B \cup l(B) \subset \bar{B}$ and (2): $\bar{B} \subset B \cup l(B)$: \\
(1): Take $x \in B \cup l(B)$, if $x \in B$ then, $x \in \bar{B}$ by definition so let $x \in l(B)$, and suppose $x \not\in \bar{B}$ for contradiction. Now $x in \bar{B}^c$ which is an open set, hence $\exists \text{ open }U$ s.t. $x \in U \and U \cap \bar{B} = \emptyset$, even further $U \cap B = \emptyset$, giving us a contradiction. \\

\noindent(2): Take $x \in \bar{B}$, even further assume $x \in (\bar{B} \setminus B)$. Now let $x \not\in l(B)$, then $\exists U \text{ s.t. } U\cap B =\{x\}$ so $x \in B$. \\

\begin{Ex}
	Suppose $(X, d)$ is a metric space and $\tau_d$ is a topology generated by the open metric balls. Show that $U \in \tau_d$ if and only if $U$ is a union of open metric balls.
\end{Ex}

\begin{Sol} \end{Sol}
\noindent Let every set $U \subset X$ have property $O$ if it is a union of open balls. Every set with property $O$ is in $\tau_d$ and thus it suffices to show that $O$ sets compose a topology. Firstly $\emptyset$ is a union of no open balls and $X$ is the union of all open balls. Now union of $O$ sets is just a bigger $O$ set by definition. Now to prove that intersection of two $O$ sets $U$ and $V$ is an $O$ set, take $x \in U \cap V$, then $x$ is in the intersection of two open balls $B_1 =B_{\epsilon_1}(x_1)$ and $B_2 =B_{\epsilon_2}(x_2)$, so now define $a := min(\epsilon_1 - d(x,x_1), \epsilon_2 - d(x,x_2))$. Now, if $y \in B_a(x)$ then $d(y,x_1) \le d(y,x) + d(x,x_1) \le a$ so $y \in B_1$ and by similar argument $y \in B_2$. So $y \in  B_1 \cap B_2$ and $B_a(x) \subset B_1 \cap B_2$. Now as this is true for every $x$ we get that the intersection is a union of open balls.

\begin{Ex}
	Let $X = \mathbb{R}$ and $d_1(x,y) = |x-y|$ and $d_2(x,y) = 2|x-y|$ be metrics. Show that $\tau_{d1} = \tau_{d2}$. 
\end{Ex}

\begin{Sol} \end{Sol}
\noindent Let's show that open sets in $d_1$ and $d_2$ are the same. Take $U$ open in $d_1$, now $\forall x \in U \exists \varepsilon_1 \text{ s.t. }$ \\$B_{\varepsilon1}(x) \subset U$. Now to prove that $U$ is open in $d_2$ just take $\varepsilon_2 = \varepsilon_1/2$ and your "epsilon ball" is in $U$. 

\begin{Ex}
	Suppose $f: X \rightarrow Y$ is a continuous and $\lim_{n \rightarrow \infty}x_n = x$. Show $\lim_{n \rightarrow \infty}f(x_n) = f(x)$.
\end{Ex}
\begin{Sol} \end{Sol} 
\noindent Suppose $U \in T_x$ is an open set containing $x$, by definition of convergence $\exists N$ s.t. $[n > N] \implies x_n \in U$. Now, as $f$ is continuous, we get $f(U) \in T_y$ and $f(x) \in U$, now we want to find $N_2$ s.t. $[n > N] \implies f(x_n) \in f(U)$. So take $N_2 = N$ and plug it in. Now $[n > N_2] \implies x_n \in U \implies f(x_n) \in f(U)$, as we wanted. \\

\begin{Ex}
	Consider $\mathbb{R}$ with following weird topology. Declare $U \subset \mathbb{R}$ open if and only if $\mathbb{R} \setminus U$ is empty, finite or countably infinite(or if $U = \emptyset$). Show that this is a topology. Now let $Y = \{0,1\}$ be a set with discrete topology. Define $f: \mathbb{R} \rightarrow Y$ so that $f(x) = 0$ iff $x \le 0$. Is $f$ continuous? Suppose $x_n \rightarrow x$. Must $\{x_n\}$ be eventually constant? Is it true that $x_n \rightarrow x \implies f(x_n) \rightarrow f(x)$? 
\end{Ex}

\begin{Sol} \end{Sol}
\noindent First let's show that $T_\mathbb{R}$ is a topology. $\emptyset \in T_\mathbb{R}$ by definition and $\mathbb{R} \in T_mathbb{R}$ as $\mathbb{R} \setminus \mathbb{R} = \emptyset$, unions of open sets are open as if $\mathbb{R} \setminus U$ is finite and $\mathbb{R} \setminus V$ is finite, then the $\mathbb{R} \setminus V \cup U$ is either smaller or equal to those differences, hence open. Now for intersections: it is important to note that two open sets in this topology can't be disjoint, hence the complement of intersection will be less than the union of two countable sets, so it will be countable, hence this is a topology.\\ 
Now to prove $f$ is continuous. Let $U = \{0\}$, $U$ is closed in $\mathbb{R}$, but $U = f^{-1}(\{0\})$ which is open in $Y$, so $f$ isn't continuous. \\
If we take a sequence $x_n \rightarrow x$ in $\mathbb{R}$ it does not have to be constant at any point as we always get arbitrarily closer to $x$ as open sets in our topology are pretty big. 

\begin{Ex}
	Suppose $X$ is metrizable, $Y$ is any space and $f: X \rightarrow Y$ is a function. Show $f$ is continuous if and only if $x_n \rightarrow x \implies f(x_n) \rightarrow f(x)$.
\end{Ex}

\begin{Sol} \end{Sol}
\noindent $\implies$: already proven in exercise 7. \\
$\impliedby:$ Suppose $f$ preserves convergent sequences. We hope to show that $f$ is continuous. Suppose $B \subset Y$ is closed, and to get a contradiction suppose $A = f^{-1}(B)$ is not closed. Get a convergent sequence $a_n \rightarrow x$ in $A$ and $x \not\in A$. We know $f(a_n) \rightarrow f(x)$. Since each $f(a_n) \in B$ and since $f(a_n) \rightarrow f(x)$ we have $f(x) \in B$. Thus $x\in A$, giving us the contradiction. 

\begin{Ex}
	Show that convergent sequences have unique limits in $T_2$ spaces.
\end{Ex}
\begin{Sol} \end{Sol}
\noindent Let $x_n$ be a convergent sequence and suppose $x_n \rightarrow x$ and $x_n \rightarrow y$ s.t. $x \neq y$. Now take $U,V \in \tau $ such that $x \in U$ and $y \in V$ but $U \cap V = \emptyset$. Now $x_n \in U$ infinitely often and $x_n \in V$ infinitely often, thus $x_n \in (U \cap V)$ infinitely often, contradicting properties of $T_2$ space.

\begin{Ex}
	Show that space $X$ is $T_1$ iff $\{x\}$ is closed $\forall x \in X$.
\end{Ex}

\begin{Sol} \end{Sol}
\noindent $\implies$: Suppose $X$ is $T_1$. Then $\forall y \in \{x\}^c$  $\exists U_y \in \tau$ s.t. $y \in U_y, x\not\in U_y$. Now $\bigcup U_y$ is open and is exactly the complement of $\{x\}$, so $\{x\}$ is closed.  

\noindent$\impliedby$: Suppose $\{x\}$ is closed $\forall x \in X$, then $\{x\}^c \in \tau$ and $y \in \{x\}^c, \forall y\neq x$ and $x \not\in \{x\}^c$ so space is $T_1$.

\begin{Ex}
Let $X=\{a,b\}$ with the following open sets:$\emptyset, \{a\}, \{a,b\}$. Is $X$ $T_0$? Is $X$ $T_1$? 
\end{Ex}

\begin{Sol} \end{Sol}
\noindent $X$ is $T_0$ as we can find $U \in \tau$, such that $a \in U$, but $b \not\in U$, just take $U = \{a\}$. On the other hand $X$ isn't $T_1$ as we can't find a set in $\tau$ containing $b$ that doesn't contain $a$. 

\begin{Ex}
	Find a space that is not $T_0$.
\end{Ex}

\begin{Sol} \end{Sol}
\noindent Just take any two point space with indiscrete topology. It is not $T_0$ as we can't get $U \in \tau$ containing just one point but not the other.

\begin{Ex}(
	Check that $h: \mathbb{R} \rightarrow \mathbb{R}$ defined so that $h(x) = 3x$ is a homeomorphism.
\end{Ex} 
\begin{Sol} \end{Sol}
\noindent First let's check if $h$ is continuous. As we're working in $\mathbb{R}$, all open sets are unions of open intervals, now if we take $U \in \tau$ in the codomain, and assume $U$ is of the form $\bigcup_{i \in I}(a_i, b_i)$, then $f^{-1}(U) = \bigcup_{i \in I}(\frac{a_i}{3}, \frac{b_i}{3})$, which is obviously open. By similar argument the invers is continuous and hence $h$ is homeomorphism. 

\begin{Ex}
	Show that $(A, \tau_A)$ is a space, assuming $A \subset X$ and $\tau_A$ is a subspace topology.
\end{Ex}

\begin{Sol} \end{Sol}
\noindent To show that $(A. \tau_A)$ is a space we need to show $A \in \tau_A$ and $\emptyset \in \tau_A$. To show this we have to find $U \in \tau_X$ such that $U \cap A = A$, and $V \in \tau_X$ such that $V \cap A = \emptyset$. Let's take $U = X$, and $V = \emptyset$. Now to show that unions and intersections are in $\tau_A$. Take $U$ and $V$ to be any open sets in $X$, then we need to show that $(V \cap A)\cap (U \cap A)$ is in $\tau_A$. To do this we just need to show that this is the same as $A \cap (V \cap U)$ as this is for sure open in subspace topology. Start by taking $x \in LHS$, thus the $x$ is in both $U$ and $V$ and it is also in $A$ so it is in $RHS$. Now when we do the converse we get the closedness under intersections. Now to prove that infinite unions are in the topology we just do the same set theory manipulation.  
\begin{Ex}
	Suppose $(X, \tau_X)$ and $(Y, \tau_Y)$ are spaces, $f: X \rightarrow Y$ is a continuous and $A \subset X$. Let $(A, \tau_A)$ be the space with subspace topology. Show $f|A$ is continuous. 
\end{Ex}

\begin{Sol} \end{Sol}
\noindent Let $U \in \tau_Y$ and let $V \in \tau_X$ be $f^{-1}(U)$. Now by definition $(f|A)^{-1} = A \cap V$ and hence open in the subspace topology, so $f|A$ is continuous.

\begin{Ex}
	Is discrete topology in fact a topology? Is course topology in fact a topology?
\end{Ex}

\begin{Sol} \end{Sol}
\noindent (1): Discrete topology is a topology because $X$ and $\emptyset$ are for sure in there, and also if we take unions or intersection over subsets of $X$ we will still get a subset of $X$. \\
\noindent (2): Course topology is a topology because by definition it only contains emppty set and the whole set, so if we take a union we get the whole set back (which is in the topology) and if we take an intersection we get back the empty set. 

\begin{Ex}
	Show that the space $(X, \tau_X)$ has the discrete topology if and only if $\{x\}$ is open in $X \forall x \in X$ if and only if $\forall (Y, \tau_Y)$, and all functions $f: X \rightarrow Y$ are continuous. 
\end{Ex}

\begin{Sol} \end{Sol}
\noindent $\implies$: Let X be a space with discrete topology, thus every $U \subset X$ is open, and $\{x\} \subset X$, hence $\{x\}$ is open. Now take $v \in \tau_Y$ and any $f : X \rightarrow Y$, now $f^{-1}(V) \subset X \implies f^{-1}(V) \in \tau_X$, so $f$ is continuous. \\

\noindent $\impliedby$: Suppose any function $f: X \rightarrow Y$ is continuous. That means that if $\forall y \in Y$, $\{y\} \in \tau_Y$, $f^{-1}(y) = \{x\}$ is open, so $\{x\} \in \tau_X$. But now, any $\bigcup_{i \in I}\{x_i\} \in \tau_X$, so any subset of $X$ is open, so $X$ has discrete topology.

\begin{Ex}
	Show that the space $(X, \tau_X)$ has course topology if and only if for all spaces $(Y, \tau_Y)$ and all functions $f: Y \rightarrow X$, $f$ is continuous. 
\end{Ex}

\begin{Sol} \end{Sol}
\noindent $\implies$: Assume $(X, \tau_X)$ has course topology, and let $(Y, \tau_Y)$ be any space and $f Y \rightarrow X$ be any function. Now to show that $f$ is continuous we need to check that $X$ and $\emptyset$ have open preimages, however this is easy as $f^{-1}(X) = Y$ and $f^{-1}(\emptyset) = \emptyset$ which are open by definition of a topology. \\
\noindent $\impliedby$: Suppose $(Y, \tau_Y), (X, \tau_X)$ are spaces and any $f: Y \rightarrow X$ is continuous. Suppose $X$ has a not course topology, then we can always construct an $f_i$ that sends a closed set in $Y$ to an open set in $X$, except for $X$ and $\emptyset$ as $f^{-1}(X) = Y$ by definition and $f^{-1}(\emptyset) = \emptyset$ which are open in $Y$, so $X$ has course topology.
\begin{Ex}
	If $(X, \tau_a)$ is finer than $(X, \tau_b)$, which of the following functions is guaranteed to be continuous? $idx : (X , \tau_a) \rightarrow (X, \tau_b)$ or $jdx : (X , \tau_b) \rightarrow (X, \tau_a)$? Is either guaranteed to be homeomorphism?
\end{Ex}

\begin{Sol} \end{Sol}
Assume, $\tau_b \subset \tau_a$, now this guarantees that $idx$ is continuous, but $jdx$ isn't continuous, and thus neither is a homeomorphism.

\begin{Ex}
	If $(X, \tau_a)$ is finer than $(X, \tau_b)$, which space is likely to have more convergent sequences?
\end{Ex}

\begin{Sol} \end{Sol}
\noindent Courser topology is likely to have more convergent sequences as all convergent sequences in $\tau_a$ converge in $\tau_b$, but the converse doesn't need to hold. 

\begin{Ex}
	Suppose $(X, \tau_X)$ is a space, $A \subset X$, and $(A, \tau_A)$ has the subspace topology. Show that if $F \subset A$, then $F$ is closed in $(A, \tau_A)$ if and only if $F = A \cap C$, and $C$ is closed in $X$.  
\end{Ex}

\begin{Sol} \end{Sol}
\noindent $\implies$: Let $F$ be closed in $A$, then $F^c \in \tau_A$ and $\exists U \in \tau_X$ such that $F^c = A \cap U$, so now $F = A \cap U^c$ and $U^c$ is closed in X exactly like we wanted.

\noindent $\impliedby$: Take $F = U \cap A$ for some closed $U \in X$, then $F^c = U^c \cap A$, and since $U^c \in \tau_X$ we know $F^c \in \tau_A$, and hence $F$ is closed in A.

\begin{Ex}
  $\mathbb{R}$ with usual topology is connected.
\end{Ex}
 
\begin{Sol} \end{Sol}
Let $A \subset \mathbb{R}$ be open, nonempty and $A \neq \mathbb{R}$, now we need to show $A$ is not closed. Let $a \in A$ and take $c \not \in A$, $a < c$. Now define a set $Z = \{ x| x \in \mathbb{R}, [a,x] \subset A \}$, and let $b = sup(Z)$. By definition of $Z$, $b \not \in Z$, but $b \in \bar{Z}$, so $b \in l(Z) $. Now, $b \not\in Z \implies b \not\in A$ and $b \in \bar{Z} \implies b \in \bar{A}$, so $l(A) \not \subset A$, hence A is closed.  

\begin{Ex}
	Suppose $X$ is a space and $A \subset X$ is connected. Show that $\bar{A}$ is connected.
\end{Ex}

\begin{Sol} \end{Sol}
\noindent Take $Y \subset X = \bar{A}$, now $A$ is dense in $Y$. Take nonempty $U \subset Y$ and assume $U$ is clopen in $Y$, now we want to show $U = Y$. By definition $U$ is clopen in $A$, further, either $U \cap A = A$ or $U \cap A = \emptyset$. To show that intersection is nonempty take $x \in U$, now if $x \not\in A$, then $x \in \bar{A} \implies x \in l(A)$. Now by definition of a limit point $\forall S \in \tau$ s.t. $x \in S$, $S \cap A \neq \emptyset$, and as $U \in \tau$, we can't have $U \cap A = \emptyset$, so $U \cap A = A$. Now: $$A \subset U \implies \bar{A} \subset \bar{U} \implies Y \subset \bar{U} \implies Y = U$$ As we wanted to show.

\begin{Ex}
	Suppose $X$ is a space and let $a \in X$. Now assume that $A_i$ is connected and $a \in A_i$ $\forall i \in I$. Show that if $\bigcup_{i \in I}A_i = X$, X is connected. 
\end{Ex}

\begin{Sol} \end{Sol}
\noindent Suppose $U \subset X$ is clopen, and $U \neq \emptyset$, now by definition, $U^c$ is clopen. Assume $a \in U$, now we want to prove $X \subset U$. To do this, let's get $x \in A_i$, note that $A_i \cap U$ is clopen and $a \in A_i \cap U$, hence $A_i \cap U = A_i$, so $x \in U$. Hence, $X = U$, $\forall \text{ clopen and nonempty } U \in X$, so $X$ is connected.

\begin{Ex}
	If $X$ is a space and $a \in X$, then there is a unique component $A \subset X$, such that $a \in A$.
\end{Ex}

\begin{Sol} \end{Sol}
\noindent Take $a \in X$ and let $A_i$ be the union of all connected subsets $B$ such that $a \in B$. Now $A_i$ is connected. Furthermore $A_i$ is maximal by it's definition. Now if $A$ and $B$ are components and $A \cap B \neq \emptyset$, then $A \cup B$ is connected, hence $A = B$. Thus the component containing any $x \in X$ is unique.

\begin{Ex}
	Suppose $(X, \tau_X)$ is connected and $f: (X, \tau_X) \rightarrow (Y, \tau_Y)$ is a continuous surjection. Show that $(Y, \tau_Y)$ is connected.
\end{Ex}

\begin{Sol} \end{Sol}
\noindent Take a clopen, nonempty $U \subset Y$ and show that $U = Y$. Assume opposite for contradiction, now $f^{-1}(U)$ and $(f^{-1}(U))^c$ are both open, but the only nonempty set satisfying that is $X$, so now $f^{-1}(U) = X$, and as $f$ is surjective that means that $U = Y$. So $Y$ is connected.

\begin{Ex}
	Suppose $X$ is a space and $A \subset X$ is a component of $X$. Why is $A$ closed?
\end{Ex}

\begin{Sol} \end{Sol}
\noindent By definition $A$ is a maximal connected set and is therefore clopen, and any clopen set is closed.

\begin{Ex}
	Suppose $X$ is a space and $A \subset X$ is closed, $B \subset X$ is closed and $f: A \rightarrow Y$ and $g: B \rightarrow Y$ are continuous such that $f|A \cap B = g|A\cap B$. Then $h = f \cup g: A\cup B \rightarrow Y$ is continuous.
\end{Ex}

\begin{Sol} \end{Sol}
\noindent Firstly, let $K \subset Y$ be closed, then $h^{-1}(K) = (h^{-1}(K) \cap A) \cup (h^{-1}(K) \cap B) = h^{-1}(K) \cap (A \cup B)$. So now $h^{-1}(K)$ is the union of two closed subsets of $X$. Thus by subspace topology, $h^{-1}$ is closed in $A \cup B$.

\begin{Ex}
	Suppose $(Y, \tau_Y)$ is a space and $X \subset Y$. Show that the following are equivalent. The subspace $(X, \tau_X)$ is compact. If $\{V_i\}$ is a collection of open sets in $\tau_Y$ covering  $X$, there exists finitely many sets $\{V_1, V_2, \dots ,V_n\} \subset \{V_i\}$ such that $V \subset \bigcup_{j=1}^n V_j$.

\end{Ex}

\begin{Sol} \end{Sol}
\noindent $\implies$: If $X$ is compact then every open cover yields a finite subcover. Now take $\{V_i\}$ to be the open cover and by definition it will yield a finite subcover, i.e $\{V_1, V_2, \dots ,V_n \}$. Converse is entirely similar. 

\begin{Ex}
	$[0,1]$ is compact using the subspace topology of $\mathbb{R}$.
\end{Ex}

\begin{Sol}\end{Sol}

\noindent Since every open $U$ in $\mathbb{R}$ is a union of open intervals, it suffices to prove the special case when $[0,1]$ is covered by open intervals. Suppose $[0,1] \subset \bigcup_{i \in I}(a_i,b_i)$. \\ Now let $K = \{x \in [0,1]| [0,x] \text{ can be covered by finitely many }(a_i, b_i) \}$. $0 \in K \implies K \neq \emptyset$. If $x \in K$ and $0 < y < x$, then $y \in K$, and since $K \subset [0,1]$, we can express $K$ as either $[0,b]$ or $[0,b)$. Now let $b$ be $sup(K)$, so $b \in K$, now if we take $b<1$, we can always find another $a \in K$ such that $a > k$, contradicting maximality of $b$, hence $b = 1$, $K = [0,1]$, $[0,1]$ is compact.

\begin{Ex}
	Suppose $(X, \tau_X)$ is a compact space, and $f: (X, \tau_X) \rightarrow (Y,\tau_Y)$ is a continuous surjection. Show that $(Y, \tau_Y)$ is a compact space.  
\end{Ex}

\begin{Sol} \end{Sol}
Take a family  of open covers $O \in \tau_Y$ covering $Y$ and prove that it yields a finite subcover. Firstly note that $f^{-1}(O) \in \tau_X$ and thus yields a finite subcover, thus the image of that finite subcover is open in $Y$ and covers $Y$. so $Y$ is compact.

\begin{Ex}
	Supoose $(X , \tau_X)$ is a compact space and $A \subset X$ is closed. Prove that $A, \tau_A$ is compact.
\end{Ex}

\begin{Sol} \end{Sol}
Cover A by open sets from $X$ and throw in the open set $V = X \setminus A$, that gives us an open cover of $X$, now take the finite subcover guaranteed by compactness of $X$, and throw out $V$, notice we have covered $A$.

\begin{Ex}
	Let $X = \{1,2,3, \dots\}$ with a funny topology. Let's call it the closed finite topology, a topology where every finite set is closed. Which of the $T_1, T_2$ axioms are true? Do convergent sequences have unique limits? Is $X$ compact? Are compact subsets closed? 
\end{Ex}
\begin{Sol}\end{Sol}
\noindent Let's check for $T_1$ first, as only finite sets are closed, this is pretty easy to confirm as for two distinct points $x,y \in X$, we can take $U_x = X \setminus \{y\}$ and $U_y = X \setminus \{x\}$, both will be open as they're infinite and they satisfy the $T_1$ axiom. \\
 Now for $T_2$, we firstly take sets $E,O \in \tau_X$ such that $E$ contains all even numbers and $O$ contains all the even ones, it is obvious that $E \cap O = \emptyset$, now for any two distinct points $x, y$, take $U_x = E \setminus \{y\} \cup \{x\}$ if $y \in E$ or $U_x = E \cup \{x\}$ otherwise, now for $U_y$ take $U_y = O \setminus \{x\} \cup \{y\}$ if $x \in O$ or $U_y = O \cup \{y\}$ otherwise. Obviously $U_x$ and $U_y$ are open and disjoint, hence satisfying the properties of $T_2$ space. Now by definition convergent sequences have unique limits. \\
  Now to check if $X$ is compact. Let $K = \{x \in X| \{1, \dots, x \} \text{ can be covered by a finite amount of open sets}\}$, now show that $K = X$. As long as $\{1,2, \dots, x\}$ is finite, it is compact, so now $K$ is unbounded, so $K = X$.
  
\begin{Ex}
	Show that $W \subset X \times Y$ is open iff $W$ is a union of sets of the form $U_i \times V_i$ with $U_i$ open in $X$ and $V_i$ open in $Y$.
\end{Ex}

\begin{Sol} \end{Sol}
\noindent $\implies$: Let $W$ be open in $X \times Y$, now by definition of topology $W$ is either a cartesian product $U \times V$ where $U \in X$, $V \in Y$ or the union of such sets.

\begin{Ex}
	Given functions $f: W \rightarrow X \text{ and } g: W \rightarrow Y$ define $(f,g): W \rightarrow X \times Y$ as $(f,g)(w)=(f(w),g(w))$. Show that $(f,g)$ is continuous if and only if $f$ and $g$ are continuous.  
\end{Ex} 
\begin{Sol} \end{Sol} 
\noindent $\implies$: Assume $(f,g)$ is continuous, that means that if $U$ is open in $X \times Y$, then $f(U)$ is open in $X$ and $g(U)$ is open in $Y$ as by the definition of product topology, so in order for $(f,g)^{-1}(U)$ to be open in $W$, then $f^{-1}(U)$ and $g^{-1}(U)$ have to be open in $W$, thus $f$ and $g$ are continuous. \\
\noindent $\impliedby$: Assume $f$ and $g$ are continuous and let $U, V$ be open in $X, Y$ respectively. Now $Z = U \times V$ is open in $X \times Y$, and $f^{-1}(U), g^{-1}(V)$ are open in $W$, and by definition of $(f,g)$, $(f,g)^{-1}(Z)$ is open in  $Z$ so $(f,g)$ is continuous.

\begin{Ex}
	Note that sets of the form $V_{i1} \cap V_{i2} \cap \dots \cap V_{in}$ must be open. Show that any open set $V \subset \prod_{i \in I}X_i$ is a union of sets of the previous format.
\end{Ex}  

\begin{Sol} \end{Sol} 
\noindent Not important for this exercise but what does $V_{i_{1}}\cap
V_{i_{2}}...\cap V_{i_{n}}$ look like? Ignoring the order that we write down
the indices, it should be of the form $U=U_{i_{1}}\times U_{i_{2}}....\times
U_{i_{n}}\times \Pi _{i\neq i_{k}}X_{i}.$

Formally $U=\Pi _{i\in I}U_{i}$ so that $U_{i}\subset X_{i}$ open in $X_{i}$
, and with finitely many exceptions $U_{i}=X_{i}.$ Such a $U$ is a \textbf{
	basic open set }$\ $\ (every open set in $\Pi X_{i}$ is a union of sets of
the format $U=\Pi _{i\in I}U_{i}$).

By the proof that follows every open set in $\Pi X_{i}$ is a union of sets
of the format $U_{i_{1}}\times U_{i_{2}}....\times U_{i_{n}}\times \Pi
_{i\neq i_{k}}X_{i}.$

(From subbasis to basis to topology.)

Notice a much more general claim. Suppose $X$ is any set and $S\subset 2^{X}$
is a collection of sets so that $\emptyset \in S$ and $X\in S$ and so that
if $\{V_{1},V_{2},..,V_{n}\}\subset S$ then $\cap _{i=1}^{n}V_{i}\in S.$
What topology does $S$ generate? Let $\tau _{S}$ be the collection of all
sets $U\subset X$ so that $U$ is a union of sets in $S.$ Keep in my we are
trying to understand the smallest (coursest) topology $\tau _{X}$ so that $%
S\subset \tau _{X}.$ Notice $\tau _{S}\subset \tau _{X}.$

$\emptyset \in \tau _{S}$

$X\in \tau _{S}$

$\tau _{S}$ is closed under unions (if for each $i\in I$ the set $U_{i}$ is
a union of sets in $S.$ Then $\cup _{i\in I}U_{i}$\ is a union of sets in $S$
why? if $x\in \cup _{i\in I}U_{i}$ then $x\in U_{i}$ for some $i$ and hence $%
x\in A\subset U_{i}$ for some $A\in S$).

Is $\tau _{S}$ closed under finite intersections

It is adequate to show that if $U\in \tau _{S}$ and $V\in \tau _{S}$ then $%
U\cap V\in \tau _{S}.$

$U=(\cup _{i\in I}A_{i})$ with $A_{i}\in S.$ And $V=(\cup _{j\in J}B_{j})$
with $B_{j}\in S.$

$U\cap V=(\cup _{i\in I}A_{i})\cap (\cup _{j\in J}B_{j})=\cup _{I\times
	J}(A_{i}\cap B_{j})$ . The last set is $\tau _{S}$ since each $A_{i}\cap
B_{j}\in S.$ (This proof is from the notes obviously i just wanted to have it handy as i liked it)
\begin{Ex}
	Take a function $f: W \rightarrow \prod_{i \in I}X_i$ and show it is continuous iff each function $\pi_i : W \rightarrow X_i$ is continuous.  
\end{Ex}   

\begin{Sol}\end{Sol}
\noindent $\implies$: Suppose $f$ is continuous and construct $id_j : \prod_{i \in I}X_i \rightarrow X_j$, by definition each $id_j$ is continuous. Now note that $f \circ id_j = \pi_j$ for each $j$, and as composition of two continuous functions is continuous, each $\pi_j$ is continuous.  \\
\noindent$\impliedby$: Suppose each $pi_i$ is continuous. Let $S= \prod_{i \in I}U_i$ be open in $\prod_{i \in I}X_i$, then $f^{-1}(S) = \bigcap_{i \in I}\pi_i^{-1}(U_i)$, a countable intersection of open sets, hence countable.

\begin{Ex}
	Show what are convergent sequences in the countable product $X_1 \times X_2 \times \dots$.
\end{Ex}
\begin{Sol}\end{Sol}
\noindent My best guess is that convergent sequences in countable product $\prod_{i \in I}X_i$ are sequences $x_n$ such that $\pi_i(x_n)$ converges in $X_i$. \\ 
$\implies$:  

\begin{Ex}
	If $X_1$ is not compact, prove that $X_1 \times X_2 \dots$ is not compact.
\end{Ex}
\begin{Sol}\end{Sol}
\noindent Suppose that the product is compact for contradiction, then for every open cover $O$ in the product, we can find a finite subcover $O_\lambda$, as $O_\lambda$ is open, it is a union of open sets, and thus its components are open in each $X_i$, so $O_1$ but  can't be finite in $X_1$, thus $O$ doesn't yield an open cover, so the product is not compact.

\begin{Ex}
	Show that if each $X_n$ is sequentially compact, then the countable product $\prod_{i=1}^{n}X_i$ is sequentially compact.
\end{Ex}  

\begin{Sol}\end{Sol}
\noindent We consider a sequence of sequences, where each is a subsequence of the last. Define $x_0 = a_1,a_2,\dots$ to be our original sequence, and construct $x_1=a_11,a_12$ to be a sequence where our first component converges(guaranteed by sequential compactness). \\
Now similarly build $x_n$ to be a subsequence of $x_{n-1}$ such that the $n$-th component converges. Now consider: \\ \\

\begin{array}{cccc}
	$a_{11}$ & $a_{12}$ & $a_{13}$ & $\cdots$ \\ 
	$a_{21}$ & $a_{22}$ & $a_{23}$ & $\cdots$ \\ 
	$a_{31}$ & $a_{32}$ & $a_{33}$ & $\cdots$ \\ 
	$\vdots$ & $\vdots$ & $\vdots$ & 
\end{array} \\\\


More specifically consider the sequence on the diagonal of the "matrix", each $a_mm$ converges in the $X_m$ and thus the whole sequence converges in the product and it is the subsequence of $x_0$, hence the product is sequentially compact. 

\begin{Ex}
	if $X \times Y$ are spaces using the product topology, define $f: X \times Y \rightarrow X$ as $f(x,y)=x$. Which among the following are guaranteed? The function $f$ is continuous? The function $f$ is open and closed? 
\end{Ex}
\begin{Sol}\end{Sol}
\noindent None of these is guaranteed as open set $U \subset X$ can have multiple elements in it's preimage, some of which will be neither open nor closed if $V \subset Y$ is closed. 

\begin{Ex}
If $X \times Y$ is compact, is $X$ compact?
\end{Ex}
\begin{Sol}\end{Sol}
\noindent By previous exercise, we know that if one component is not compact, the whole product is not compact. So in order for the product to be compact we need $X$ to be compact.

\begin{Ex}
	Prove that every metric space is Hausdorff($T_2$).
\end{Ex}
\begin{Sol}\end{Sol}
\noindent Let $X$ be a metric space with metric $d$. Get two distinct points $x,y \in X$, To find $U$ containing $x$ but not $y$ just let $U$ be the open ball with radius $\frac{d(x,y)}{2}$ centered at $x$, and do the same for $y$. 

\begin{Ex}
	If $X$ is $T_2$ and $Y$ is $T_2$, prove that $X \times Y$ is $T_2$.
\end{Ex}
\begin{Sol}\end{Sol}
\noindent Take points $(x,y)$ and $(a,b)$ in $X \times Y$, now if we take a set $U \subset X$ such that $x \in U$ but $a \not \in U$, existance of $U$ is given to us by $X$ being $T_2$ and we find a similar $V \subset Y$ for $y$ and $b$, $U \times V$ will be open in product space and $(x,y) \in U \times V$ but $(a,b) \not \in U \times V$. now we can find the same set containing $(a,b)$, thus $X \times Y$ is $T_2$. Is $X$ compact, connected, locally connected.

\begin{Ex}
	Since $[0,1]$ is compact and metrizable, and since the finite product $[0,1] \times \dots \times [0,1]$ is compact and $T_2$, what can you say about the relationship between the closed and compact subsets of $[0,1] \times \dots \times [0,1]$?   
\end{Ex}

\begin{Sol}\end{Sol}
\noindent \textbf{Claim: }Every closed subset of $[0,1] \times \dots \times [0,1]$ is compact. To prove this just note that if we take a closed subset $V$, we get a free open set $V^c$ now take any open cover of $V$ and add $V^c$ in order to cover the whole product, now by compactness you get a free finite subcover of the product, and you have covered $V$. \\
\textbf{Claim: }Every compact subset of $[0,1] \times \dots \times [0,1]$ is closed. To prove this we will assume that a compact subset $U$ is open for contradiction. Then the set is of the form $(a_1,b_1) \times (a_2,b_2) \times \cdots \times (a_n,b_n)$, but by previous exercise, if any of the multiples is not compact, the product is not compact, so we just need to prove that $(a,b) \subset [0,1]$ is not compact. To do that take the open cover made by open intervals $(0,\frac{1}{n})$, $\frac{1}{n}\le b$, this doesn't yield a finite subcover Thus, every compact subset is closed.

\begin{Ex}
	According to a legend, the compact subsets of the Eucledian metric space $\mathbb{R}^n$ are preciesly the closed and bounded subsets of $\mathbb{R}^n$. Prove this.
\end{Ex}

\begin{Sol} \end{Sol}
Let $X$ be closed and bounded in $\mathbb{R}^n$, then $X = \prod_{i=1}^{n}[a_i,b_i]$, now every $[a_i,b_i]$ is compact and we know that product of two compact sets is compact, so by simple induction the whole countable product is compact so closed bounded sets are compact in $\mathbb{R}^n$. \\
For converse assume $X \subset \mathbb{R}$ is compact but for contradiction suppose it is open, then it is of the form $(a_1,b_1) \times (a_2,b_2) \times \cdots $, and as we previously established, we need only one component of the product to fail in order for product to fail to be compact, and open intervals aren't compact in $\mathbb{R}$ which finishes the proof.

\begin{Ex}
	Suppose $X$ is compact and $f: X \rightarrow Y$ is continuous and $Y$ is $T_2$. Prove $f$ is a closed map.
\end{Ex}  

\begin{Sol}\end{Sol}
\noindent Take a closed $C \subset X$, we want to prove $f(C)$ is closed. First note that $C$ is compact as it is a subset of a compact space. Hence $f(C)$ is a compact subset of a $T_2$ space. Now to prove that $K = f(C)$ is closed take $y \in Y\setminus K$, since $Y$ is $T_2$ there are disjoint open sets $U_y$ and $V_z$ such that $y\in U_y$ and $z\in V_z$ for every $z\in Y$. Now $\bigcup_{i\in I} V_{zi}$ is an open cover of $K$ and yields a finite subcover $\{V_z | z \in L\}$ where $L$ is finite. Now $\bigcap_{z \in F}U_y$ is an open neighborhood of $y$ disjoint from $K$. Since $y$ is arbitrary $K$ is closed.   

\begin{Ex}
	If X is metrizable and A is not closed. Then if $x \in \bar{A}\setminus A$ there exists a sequence $a_n$ converging to $x$ and for all $n$ we have $a_n \in A$
\end{Ex}

\begin{Sol}\end{Sol}
\noindent Get a metric $d$ such that every open set in $X$ is a union of open metric balls. Get the wanted $x \in \bar{A} \setminus A$. $x$ is a limit point of $A$. For each $n\in\mathbb{N}$ obtain $a_n \in A$ such that $a_n \in B(x, \frac{1}{2^n})$. Now $x$ is in said open ball  because $d$ is a metric and $x$ is in every open ball around $x$. Note that $x$ is a limit point of $A$ but $x\not \in A$ and the open ball is open and contains $x$ therefore the mentioned $a_n$ exists.\\
Now to show that $a_n$ converges to $x$. Suppose $U$ is and open set containing $x$. We must find N such that if $n\ge N$ then $a_n \in U$. Since $x\in U$ and $U$ is open, $U$ is a union of open balls of $d$. Thus, there exists $y\in X$ and $\epsilon \ge 0$ such that $x$ in open $\epsilon$-ball around around $y$, call it $B_y$, note that $B_y \subset U$. Thus $d(x,y) < \epsilon$. Now take $\delta = \epsilon - d(x,y)$. Get N such that if $n>N$ then $d(x,a_n) < \delta$. Now when $n\ge N$ we get $d(a_n,y) \le d(a_n,x)+d(x,y)< \delta + (\epsilon-\delta)=\epsilon$. Thus $a_n$ converges to $x$.

\begin{Ex}
	Suppose $X$ is any space and $B \subset X$ is closed. Suppose $b_n \in B$ for each $n$ and suppose the sequence $b_n \rightarrow x$. Then $x \in B$.
\end{Ex}

\begin{Sol}\end{Sol}
Suppose $x\not in B$. Then $x$ is not a limit point of $B$. Let $U$ be open and contain $x$. Since $b_n \rightarrow x$, get $N$ such that $n>N \implies b_n \in U$. Since $b_n \in B$ and $x \not \in B$, this means that $x$ is a limit point of $B$ yielding us a contradiction.

\begin{Ex}
	If $X$ is path connected then $X$ is connected
\end{Ex}
\begin{Sol}\end{Sol}
\noindent Fix $a \in X$, now get a path $\alpha_b$ for each $b\in X$. Now notice that image of every $\alpha_b$ is connected and thus the union of all images is connected, that union is the space $X$ and thus $X$ is connected.

\begin{Ex}
	If $X$ is nonempty, connected and locally path connected then $X$ is path connected.
\end{Ex}
\begin{Sol}\end{Sol}
\noindent Fix $a\in X$. Defne $U_a = \{x \in X|\exists a path \alpha:[0; 1] \rightarrow X$ so that
$\alpha(0) = a and \alpha(1) = x\}$. Hope to show: $U_a = X$. Obviouslt $U_a$ is nonempty as constant path $\alpha_a$ exists so $a \in U_a$. Now  to show that $U_a$ is open: fix $x \in U_a$ and choose a subset $U$ containing $x$. Now all $u \in U$ are path connected to $x$ and thus path connected to $a$ so $U \subset U_a$ and $U_a$ is open. To show that $U_a$ is closed look at the closure of $U_a$. Look at $y\in \bar{U_a}$, and choose an open path connected set $V$ containing $y$. Now as $V \cap U_a \neq \emptyset$ hence take $z$ in the intersection which is path connected to $y$ and and thus $y$ is path connected to $a$, so now $U_a = \bar{U_a}$, thus $\bar{U_a}$ is closed. Now as $U_a$ is clopen and nonempty so it is actually our space $X$ and thus our space is path connected.

\begin{Ex}
	Consider the following subset $X \subset \mathbb{R}^2$. Let $X = [(0,0),(0,1)] \cup [(0,0),(1,0)] \cup [\bigcup_{n=2}^{\infty}(\frac{1}{n},0),(\frac{1}{n},1)]$. Is $X$ compact? Connected? Locally connected? 
\end{Ex}

\begin{Sol}\end{Sol}
Can't think about this right now it seems hard, will do it later

\begin{Ex}
	Prove that $T_4 \implies T_3 \implies T_2$.
\end{Ex}
\begin{Sol}\end{Sol}
Suppose $X$ is $T_4$, then for each disjoint pair of closed sets $A$ and $B$ we can find disjoint pair of open sets $A\subset U$ and $B\subset V$, so now in order for our set to be $T_3$ we will take $a \in A$ and keep the set $U$ and now $a$ and $B$ suffice the definition of $T_3$. To further this into a $T_2$ argument just take $b \in B$ while keeping the open set $V$ and we're done.

\begin{Ex}
	Suppose $X$ is compact and $T_2$. Show that $X$ is $T_3$.
\end{Ex} 
\begin{Sol}\end{Sol}
\noindent	Let $X$ be a compact $T_2$ space, it is obviously compact so in order to show it is $T_3$ we just need to show that it is regular. Take $x \in X$ and let $B$ be a closed set in $X$ not containing $x$, $B$ is compact. Since $X$ is $T_2$ we can find $U_x$ and $U_b$ for every $b \in B$ such that $U_x \cap U_b = \emptyset$. Now consider the cover of $B$: $\mathcal{B}=\{U_b|b\in B \}$, by compactness of $B$ we can reduce this to a finite subcover $\mathcal{B}' = \{U_{b1},U_{b2}, \dots, U_{bn} \}$. Now let $V = \bigcup_{i=1}^n U_{bi}$, it is clearly open and contains $b$ but not $x$. Now let $U = \bigcap_{i=1}^n U_{xi}$, $U$ is open and contains $x$ but not $b$ and thus $U \cap V = \emptyset$. So now we have satisfied regularity and $X$ is $T_3$.
	
\begin{Ex}
	Suppose $X$ is compact and $T_2$. Show that $X$ is $T_3$.
\end{Ex}

\begin{Sol}\end{Sol}
\noindent Let $X$ be a compact $T_2$ space, then for $T_4$ we just need to show normality of $X$ as it is already $T_1$. Take disjoint closed sets $A$ and $B$ in $X$. We have proven that $X$ is $T_3$ so, for every $a \in A$ exists $U_a$ and $V_a$ satisfying the properties of $T_3$. Now consider the following open cover of $A$: $\mathcal{A} = \{U_a | a \in A\}$, clearly this open cover yields a finite subcover $\mathcal{A}' = \{U_{a1},\dots,U_{an} \}$. Now let $U$ be the union of sets in $\mathcal{A}'$, $A \subset U$. Now let $V$ be the finite intersection of coresponding $V_{xi}$. $V$ is open in $X$ and $B \subset U$, moreover by the construction $U \cap V =\emptyset$ satisfying the properties of $T_4$.

\begin{Ex}
	Suppose $X$ is metrizable, show that $X$ is $T_3$
\end{Ex}
\begin{Sol}\end{Sol}
\noindent Suppose $a \in X$ and take closed $B$ not containing $a$. Get a $d$ metric on $X$. Now there exists $\epsilon > 0$ such that $B_\epsilon(a) \cap B = \emptyset$, then  let $ U= B_{\frac{\epsilon}{2}}(a)$ and let $V= \bigcup_{b\in B} B_{\frac{\epsilon}{2}}(b)$. Clearly $U\cap V = \emptyset$ sufficing conditions of $T_3$.

\begin{Ex}
	Suppose $X$ is metrizable, show that $X$ is $T_4$
\end{Ex}

\begin{Sol}
\noindent Let $A$ and $C$ be disjoint closed subsets in $X$. Now for each $x \in A$ find $\epsilon_x$ such that open metric ball $B(x,\epsilon_x)$ is disjoint with $C$, similarly find $\epsilon_y$ for every $y \in C$. Now let $U = \bigcup_{i \in I}B(y_i, \epsilon_{yi}/3)$ and $V = \bigcup_{i \in I}B(x_i, \epsilon_{xi}/3)$. Clearly, $V$ and $U$ are open in $X$, and $A \subset V$, $C \subset U$. \\
Now suppose $z \in U \cap V$, then $d(x,z)< \frac{\epsilon_x}{3}$ and $d(y,z) < \frac{\epsilon_y}{3}$. That would imply that $d(x,y) \le \frac{\epsilon_y}{3} + \frac{\epsilon_x}{3} < \epsilon_x$, assuming $\epsilon_x > \epsilon_y$. But then $y \in B(x,\epsilon_x)$, yielding a contradiction.
\end{Sol}

\begin{Ex}
	Suppose $X$ is compact, show that $X$ is locally compact.
\end{Ex}

\begin{Sol}\end{Sol}
\noindent $X$ is open and is an open neighborhood of every point in the space, and as $\bar{X}= X$ it's closure is compact, so $X$ is locally compact.
\begin{Ex}
	Suppose each of $X$ and $Y$ are locally compact. Prove that $X \times Y$ is locally compact.
\end{Ex}

\begin{Sol}\end{Sol}
\noindent Let $(x,y)$ be a point in $X \times Y$, now as $X$ is locally compact, there exists an open $U$ containing $X$ such that $\bar{U}$ is compact, similaryly we find an open $V$ with compact closure in $Y$. Now $U \times V$ is open in the product topology and$\bar{U} \times \bar{V}$ is compact in the product space as it is a product of compact sets, there fore we found our open set with compact closure and $X \times Y$ is locally compact. 

\begin{Ex}
	Show that countable product $\R \times \R \times \cdots \times \R$ is not locally compact.
\end{Ex}

\begin{Sol}\end{Sol}
\noindent Consider a point $x^n = (x_1,x_2,\dots, x_n)$ in the countable product. Every open set $U$ containing $x^n$ is of the form $(a_1,b_1)\times(a_2,b_2)\times \cdots \times \R \times \R$ so it's closure won't be compact as $\R$ is not compact, thus sufficing to show the countable product is not compact.

\begin{Ex}
	Suppose $X$ is linearly ordered space with at least two elements $a < b$. Suppose $m$ is minimal in $X$ and $m < x\in X$. Must the half open interval $[m,x)$ be open? Suppose $M$ is maximal in $X$. If $x<M$ must the half open interval $(x,M]$ be open in $X$? Suppose $X$ has only one point. What are the open rays? What are the open sets in $X$?
\end{Ex}
\begin{Sol}\end{Sol}
\noindent Half open interval $[m,x)$ is actually just left open ray $L_x$ and is therefore open in $X$. Similarly $(x,M]$ is $R_x$. If $X$ has only one point $x$, then $L_x = R_x = x$ and only open sets are $X$ and $\emptyset$.

\begin{Ex}
	Does every linearly ordered space $(X,<)$ satisfy the $T_2$ axiom?
 \end{Ex}

\begin{Sol}\end{Sol}
\noindent Suppose $a,b \in X$ and $a \neq b$, then $\{a,b\}\subset X$. Assume $a<b$, then we have two cases: \\
1) If there exists $c$ such that $a<c<b$, then we can just take $L_c$ and $R_c$ and thus satisfy the $T_2$ axiom\\
2) If there is no point between $a$ and $b$, then just take $R_a$ and $L_b$ and satisfy the $T_2$ axiom.

\begin{Ex}
	Suppose $(X,<)$ is a well ordered space with $x_1 \ge x_2 \ge x_3 \dots$. Must $\lim_{n \rightarrow \infty}$ exist? Is it possible that all terms of $x_n$ are distinct?
\end{Ex}

\begin{Sol}\end{Sol}
\noindent The limit must exist as we have a minimal element and our sequence has to become constant at some point, thus not all entries can be distinct.

\begin{Ex}
Suppose $(X,<)$ is an infinite well ordered space with maximal element $M$ and suppose $x_n$ is a sequence in $X$. Show that $x_n$ has a constant subsequence $ x_{n1} \le x_{n2} \le \cdots $. Must $x_{mj}$ converge? 
\end{Ex}

\begin{Sol}
	\noindent If there exists a constant subsequence it immediately solves both of our problems. So let's assume such a subsequence doesn't exist. For starters take only distinct terms of $x_n$. Then let $x_{n1} =m_1 = min\{x_1,x_2,\dots \}$. Let $m_2 = x_{n2}$ be the minimum of the terms with index bigger than $n_1$ and so on. This creates a subsequence $m_1 < m_2 < \dots$  \\ 
	Now we have $m_n < M$ for all $n$. Let $B = \{ x \in X|m_n < x \forall n \}$. Let $b=min(B)$. Now clearly $m_n < b$, so let $U$ be an open set containing $b$. If $b\in L_x$ for some $x$ then we have $m_n \in L_x$. If $b \in R_x$ then $x\not \in B$ and hence there exists some n for which $x<m_n$. Thus $x<m_{n+k}$, this shows that $m_n \rightarrow b$.
\end{Sol}

\begin{Ex}
	Suppose $X$ is a space and $Y$ is a set and $f : X \rightarrow Y$ is a surjection. Declare $U \subset Y$ open iff $f^{-1}(U)$ is open. Show this is really a topology on $Y$.  
\end{Ex}
\begin{Sol}\end{Sol}
\noindent	First ti show that the set $Y$ is open in $Y$, it is obvious it is as $f^{-1}(Y)=X$. Similarly $\emptyset$ is open in $Y$ as it is its own preimage. So know we have to show that if $U_1, U_2, \dots $ are open in $Y$ so is their union. For this just take in consideration that $f^{-1}(\bigcup_{i \in I}U_i) = \bigcup_{i \in I} f^{-1}(U_i)$ and as every individual preimage is open so will be the union of preimages and thus our union is open. Now we want to show that $U \cap V$ is open assuming both $U$ and $V$ are open. To prove this just consider that $f^{-1}(V \cap U) = f^{-1}(U) \cap f^{-1}(V)$ and as both preimages are open so is their intersection and thus $U \cap V$ is open finishing the proof.

\begin{Ex}
	Every closed map is a quotient map
\end{Ex}
\begin{Sol}\end{Sol}
\noindent Take spaces $X$ and $Y$ and a closed map $f: X \rightarrow Y$. Now if we take a closed set $K$ in $Y$, we know that $f^{-1}(K)$ is closed and thus $(f^{-1}(K))^c$ is open and mapped to $K^c$ and thus open sets are mapped to open sets and our map is a quotient map. 

\begin{Ex}
	Every open map is a quotient map
\end{Ex}

\begin{Sol}\end{Sol}
\noindent Take spaces $X$ and $Y$ and open map $f$, then any open set $U$ in $Y$ has open preimage $V$ in $X$, similarly any open set $V$ is mapped to an open $U$ finishing the proof.

\begin{Ex}
	Every homeomorphism is a quotient map.
\end{Ex}

\begin{Sol}\end{Sol}
\noindent Take to spaces $X$ and $Y$ and homeomorphism $h$. Now $h$ being continuous guarantees that preimage of every open $U$ in $Y$ is open in $X$. Now as $h$ is homeomorphic that gives us continuous $h^{-1}$ which guarantees us that each open $V$ in $X$ is mapped to an open set in $Y$, hence open sets in $Y$ are exactly the ones whose preimage is open.    

\begin{Ex}
	Being homotopic is an equivalence relation on the set of maps $C(X,Y)$.
\end{Ex}
\begin{Sol}\end{Sol}
\noindent 1) Is $f~f$? Define $H: X \times [0,1] \rightarrow Y$ as $H(x,t) = f(x)$ \\
2) Suppose $f \sim g$, is $g~f$? Let $G_t = H_{1-t}$ if $H$ connects $f$ to $g$ \\
3) Suppose $f \sim g$ and $g \sim h$. Given the homotopies $H_{fg}$ and $H_{gh}$ and define $H_{fh}(t)$ as $H_{fg}(2t)$ for $t \in (0,\frac{1}{2})$ and $H_{gh}(2t-1)$ for $t\in (\frac{1}{2},1)$.    

\begin{Ex}
	Homotopy equivalence is an equivalence relation.
\end{Ex}

\begin{Sol}\end{Sol}
\noindent 1) Is $X\sim X$? Let $f = g = id_X$. Define $H: X \times [0,1] \rightarrow X$ as $H(x,t)=x$ \\
2) If $X \sim Y$, is $Y \sim X$? True by definition of homotopy inverses \\
3) If $X \sim Y$ and $Y \sim Z$ is $X \sim Z$? Start with $f: X \ra Y$, $g : Y \ra X$ and $h: Y \ra Z$, $k: Z \ra Y$. \\ 
Now we get $hf: X \ra Z$ and $gk: Z \ra X$

\begin{Ex}
	Show that $X$ is contractible if and only if $X$ is homotopy equivalent to a one point space.
\end{Ex} 

\begin{Sol}\end{Sol}
\noindent $\implies$: Assume $X$ is contractible, that means that $id_x$ is homotopic to a constant map to a one point subspace of $X$, now we want to show that $X$ And $A = \{a\}$ are homotopy equivalent. So take maps $f: X \ra A$ such that $f(x) =a$ and $g: A \ra X$ such that $g(a) = x_0$. \\
Now consider $fg : A \ra A$, this is clearly exactly $id_A$ and hence homotopic to $id_A$. Now take into consideraton $gf: X \ra X$ and notice that $gf(x) = x_0$, $\forall x\in X$, but we alredy know that $id_X$ is homotopic to this by contractability of $X$. \\
\noindent $\impliedby$: Assume $X$ and $A$ are homotopy equivalent. Take same $f$ and $g$ as in the previous part. Now as we know that $gf: X \ra X$ is actually constant map $X \ra \{x_0\}$ and it is homotopic to $id_X$ which is exactly what we want in the definition of contractability.

\begin{Ex}
	Suppose $(X,d)$ is a metric space and suppose the continuous surjection $q: X \ra Y$ is a quotient map. Show that $Y$ is a sequential space.
\end{Ex}

\begin{Sol}\end{Sol}
\noindent As we know that $X$ is a metric space, we know that $X$ is sequential, so take a nonclosed set $A$ and a sequence $a_1,a_2,\dots \ra x$ where $x \not\in A$. Now as we know $A$ is nonclosed and $q$ is a quotient map, we know that $q(A)$ is nonclosed. Now as we know that $q$ is continuous it will map converging sequences to converging sequences so $g(a_1), q(a_2) \dots \ra q(x)$ and $q(x) \not\in q(A)$ so $Y$ is sequential.

\begin{Ex}
	Suppose $X$ is a sequential space and $Z$ is any space and $f: X\ra Z$ is any map. Then $f$ is continuous iff for each sequence $\{x_1,x_2,\dots x\}$ then $f(x_n) \ra f(x)$  
\end{Ex}

\begin{Sol}
\end{Sol}
\noindent $\impliedby$: Suppose that for an arbitrary sequence $\{x_1,x_2,\dots x\}$ in $X$ we have $f(x_n) \ra f(x)$. Now find an open $K$ in $Z$,    


\begin{Ex}
	Suppose for eacth $i\in I$ that $A_i$ is a closed compact subset of the compact space $X$. Then $\emptyset \neq \bigcap_{i \in I}A_i$ if and only if $A_{i1} \cap A_{i2} \cap \cdots \cap A_{ik} \neq \emptyset$ for finite $k$.
\end{Ex}
\begin{Sol}
\end{Sol}
\noindent One direction is trivial. For the other direction let $U_i = X \setminus A_i$ for each $i$. If $\bigcap_{i \in I}A_i = \emptyset$ then $\{U_i\}$ is an open cover admiting an open subcover  of X, however,  finite union of $U_i$ would fail to cover a finite intersection of $A_i$ giving us a contradiction.

\begin{Ex}
	Show that if $dim(X) = 0$, then $dim(X \times X \cdots) = 0$.
\end{Ex}

\begin{Sol}\end{Sol}
\noindent If $dim(X)=0$ then the basis of $X$ is composed of clopen sets, so know denote the basis as $A_1,A_2,\dots$, if we consider the product topology on the product, the base will be composed of the sets of the form $\prod_{i=j}^{n}A_i \times X \times X \cdots$, and thus be clopen as each component is clopen in $X$.

\begin{Ex}
	Let $C = \{0,1\} \times \{0,1\} \dots$ Is $C\times C$ homeomorphic to $C$? 
\end{Ex}

\begin{Sol}\end{Sol}
Try $h: C \ra C \times C$ defined as $h(s_1,s_2,\dots) = (s_1,s_3,s_5,\dots)(s_2,s_4,\dots)$. To show that $h$ is injective just take $a$ and $b$ in C such that $h(a) = h(b)$, that means that all their odd and even coordinates are the same, so $a = b$. To prove it's surjective just take any $c \in C \times C$ and reconstruct the $h^{-1}(c)$. So now to finish we just have to prove $h$ is continuous and we are done. To do this notice that open sets 


\begin{Ex}
	Must every homeomorphism $h: \R \ra \R$ be such that $h(q) \in \mathbb{Q}$ for all $q \in \mathbb{Q}$? Is it true if $h$ isometry or linear.
\end{Ex}

\begin{Sol}\end{Sol}
\noindent Consider $h(x) = x +\sqrt{2}$ which is clearly injective, surjective and continuous but does not map $\mathbb{Q}$ to $\mathbb{Q}$. this $h$ is also an isometry so it works for that as well. To show that it doesn't work if $h$ is linear just consider $h(x) = \sqrt{2} \cdot x$, it does everything we need.

\begin{Ex}
	Suppose $g: A \ra B$ is a continuous bijection. Prove $g$ is a homeomorphism iff $g$ is a quotient map.
 \end{Ex} 
\begin{Sol}
\end{Sol}
\noindent One direction is obvious as $g$ being a homeomorphism implies that $g$ is a quotient map. For other direction assume $g$ is a continuous, bijective quotient map, so now we have to prove that $g^{-1}$ is continuous in order to get what we need. So let $K$ be open in $A$, so now we want to know that $(g^{-1})^{-1}(K)$ is open in $B$, but that is $g(K)$ which we know is open as $g$ is continuous.

\begin{Ex}
	Suppose $r: X \ra A \subset X$ is a retraction. Then $r(x) = x$ if and only if $x \in A$.
\end{Ex}

\begin{Sol}
\end{Sol}
\noindent If $x\in A$ then we get what we want by definition of a retraction. If $x\not \in A$ then since $im(r) \subset A$ and thus $r(x) \in A$ so $r(x) \neq x$. 

\begin{Ex}
	Suppose $X = [0,1]$ and $A = \{0,1\}$. Prove or disprove that $A$ is a retract of $X$.
\end{Ex}
\begin{Sol}\end{Sol}
\noindent Since $X$ is connected and $A$ isn't no function $X \ra A$ can be continuous.

\begin{Ex}
	Suppose $X$ is $T_2$ and $A \subset X$ and $r:X \ra A$ is a retraction. Prove that $A$ is closed.
\end{Ex}

\begin{Sol}\end{Sol}

\noindent Take $x \not \in A$, so $r(x) \in A$ and $x \neq r(x)$. So by $X$ being $T_2$ we can get disjoint open sets such that $x\in U$ and $r(x) \in V$. Let $W = U \cap r^{-1}(V)$. We know $x\in W$ so we WTS that $x \not \in A$. If $a \in U \cap A$, then then $r(a) = a \not \in V$. Now as $x \in r^{-1}(V)$ iff $r(x) \in V$ so our $a \not\in r^{-1}(V)$. This shows $W \cap A = \emptyset$.

\begin{Ex}
	$x \ra x^2$ and $x \ra |x|$. Is either a retraction? 
\end{Ex} 


\begin{Sol}
\end{Sol}
\noindent First fnction isn't a retraction as it has fixed point set $\{0,1\}$ which isn't connected so by previous exercise function can't be a retraction. Second function is a retraction with retract $A = \R^{\ge 0}$.
\end{document}